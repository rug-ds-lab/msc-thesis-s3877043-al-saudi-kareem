%*******************************************************
% Abstract
%*******************************************************
\clearpage
\phantomsection
\pdfbookmark[1]{Abstract}{Abstract}
\addcontentsline{toc}{chapter}{Abstract}

\begingroup
\let\clearpage\relax
\let\cleardoublepage\relax
\let\cleardoublepage\relax

\chapter*{Abstract}
By virtue of the steady societal shift to smart technologies, built on the increasingly popular smart grid framework, we have noticed an increase in the need to analyze household electricity consumption at the individual level. In order to work efficiently, these technologies rely on load forecasting to be able to optimize operations that are related to energy consumption (such as household appliance scheduling). This paper proposes a novel load forecasting method that utilizes a clustering step prior to the forecasting step to group together days that exhibit similar patterns in energy consumption. Following that, we attempt to classify new days into one of the pre-generated clusters by making use of the available context information (day of the week, month, predicted weather). Finally, using available historical data (with regards to energy consumption) alongside meteorological and temporal variables, we train a \gls{cnn-lstm} model on a per-cluster basis that each specializes in forecasting based on the energy profiles present within each cluster. This method leads to improvements in forecasting performance (upwards of a 10\% increase in \gls{mape} scores) and provides us with the added benefit of being able to easily highlight and extract information that allows us to assess which external variables have an effect on the energy consumption of any individual household.

\vfill

\endgroup

\vfill