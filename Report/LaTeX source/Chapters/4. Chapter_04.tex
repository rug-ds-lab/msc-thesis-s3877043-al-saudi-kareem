\chapter{Data Description}
\label{ch:Data-Description}
At our disposal are a number of publicly available data sets that contain historical data with regards to energy consumption. These include the data collected by the \gls{epsrc} via the project entitled "\textit{\gls{refit}}" \cite{REFIT} which is a collaboration among the Universities of Strathclyde, Loughborough and East Anglia and the \textit{"Individual Household Electric Power Consumption"} data set \cite{UCID} that is part of the \gls{uci} Machine Learning Repository and that will henceforth be acronymized as the \textit{"\gls{ucid}"}. This section will serve to briefly describe the main aspects of each of these individual data sets so that we may be better able to draw comparisons between them and highlight any key differences. Further in-depth analysis of each subsequent data set can be found in section \ref{ch:Exploratory-Data-Analysis} of this paper. Additionally, we aim to append meteorological features (\eg temperature, wind speed, cloud coverage, precipitation) to each of our respective data sets -- an overview of this process and the data that we will be utilizing will also be presented in this section.

\section{REFIT}
\label{sec:Data-Description:REFIT}
The \gls{refit} Electrical Load Measurements data set includes cleaned electrical consumption data, in watts, for a total of 20 households labelled \textit{House 1 - House 21} (skipping House 14) located in the Loughborough area, a town in England, over the period of 2013 through early 2015. The electrical consumption data is collected at both the aggregate level as well as the appliance level with each household containing a total of 10 power sensors that comprise of a current clamp for the household aggregate labelled as \textit{Aggregate} in the data set as well as 9 \glspl{iam} labelled as \textit{Appliance 1 - Appliance 9} in the data set. The appliance list associated with each of the \glspl{iam} differs between households and comprise a measure of ambiguity as applicants may have switched appliances around during the duration of the data collection and the installation team responsible for setting up the power sensors did not always collect relevant data associated with said \glspl{iam}. The consequences of this is of course that we do not know with 100\% certainty whether an appliance or set of appliances associated with an \gls{iam} is the same throughout the entirety of the data set. Additionally, some labels are inherently ambiguous taking, for example, the \textit{television site} label which could comprise of any number of appliances including: a television, DvD player, computer, speakers etc. Finally, the models and makes of the appliances that were meant to be collected by the installation team are not always present further compounding on the previously mentioned uncertainties.

\noindent \newline The documentation associated with the data set states that active power is collected, and subsequently recorded, at an interval of 8 seconds; however, a cursory glance at the data demonstrates that this is not always the case. A potential reason for this could be the fact that the aforementioned power sensors are not synchronised with the associated collection script which polls within a range of 6 to 8 seconds leaving us with a margin for error in the intervals between recorded data samples. Moreover, the data set is riddled with long periods of missing data making it exceptionally difficult to work with. All of that said, the data collection team made an attempt to pre-process or otherwise \textit{clean} the data set by:

\begin{enumerate}
    \item Correcting the time to account for the \gls{uk} daylight savings.
    \item Merging timestamp duplicates.
    \item Moving sections of \gls{iam} columns to correctly match the appliance they were recording when said appliance was reset or otherwise moved.
    \item Forward filling \gls{nan} values or zeroing them depending on the duration of the time gap.
    \item Removing spikes of greater than 4,000 watts from the \gls{iam} values and replacing them with 0s.
    \item Appending an additional issues columns that is set to 1 if the sum of the sub-metering \glspl{iam} is greater than that of the household aggregate -- in this case, data should either be discarded or, at the very least, the discrepancy must be noted.
\end{enumerate}

\section{UCID}
\label{sec:Data-Description:UCID}
The \gls{ucid} data set contains a total of 2,075,259 measurements gathered in a single house located in Sceaux, a commune in the southern suburbs of Paris, France. The data within this data set was recorded throughout a duration of 47 months spanning the period between December 2006 and November 2010. Measurements were made approximately once a minute and consist of the minute-averaged active power consumption, in kilowatts, within the entire household as well as 3 energy sub-metering measurements that correspond to the kitchen, which includes a dishwasher and microwave, the laundry room that consists of a washing machine and tumble dryer, and the combination of both an electric water-heater as well as an air-conditioner respectively. The \gls{ucid} data set is not without fault either containing approximately 25,979 missing measurements which make up roughly 1.25\% of the entire data set; however, given the extensive range covered as well as the immense number of total measurements available on hand these missing values can easily be disregarded and subsequently discarded during the preprocessing stage of our forecasting pipeline.

\section{Meteorological Data}
\label{sec:Data-Description:Meteorological-Data}
As an addendum to both the \gls{refit} and \gls{ucid} data sets we will be incorporating meteorological data as provided by Solcast \cite{Solcast}, a company based in Australia that aims to provide high quality and easily-accessible solar data. This service is not provided free of charge; however, public researchers and students are allotted a generous amount of credit to work with and, per request, are entitled to received additional credit as needed. For the purpose of this master's thesis project we will be requesting meteorological data in variable time resolutions (5, 10, 15 minutes) for both the Loughborough area in the \gls{uk} for the \gls{refit} data set as well as meteorological data for the Sceaux commune in the southern suburbs of Paris, France for the \gls{ucid} data set. The relevant periods are the 16th of September, 2013 up to and including the 11th of July, 2015 and the 1st of December, 2006 up to and including the 30th of November, 2010 for each data set respectively. The provided data is extensive, covering a wide range of parameters that are listed, and described in detail, in Table \ref{tab:Solcast-parameters}.