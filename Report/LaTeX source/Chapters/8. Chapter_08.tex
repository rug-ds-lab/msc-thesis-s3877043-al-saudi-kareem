\chapter{Conclusion and Future Work}
\label{ch:Conclusion-and-Future-Work}
In this study, we have shown that the application of a clustering step that utilizes dimensionality reduction techniques such as \gls{t-sne} and hierarchical, density-based clustering in the form of \gls{hdbscan} leads to significant improvements in forecasting accuracy when taking individual households into consideration. While this technique is certainly more complex, in particular with regards to the number of steps and moving parts associated with the entire pipeline, we maintain that the benefits in terms of improved forecasting accuracy outweigh the overall increase with regards to the time and effort it would take to train and set up such a model. The practicality of the model lies in the availability of the data that it requires to function -- primarily with respect to historical energy consumption data for the individual households in question (which is becoming easier and easier to obtain thanks to the prevalence of smart meters) and meteorological data that can easily be obtained from numerous sources. Furthermore, it is highly likely that, given enough historical data, the need to further train the model(s) after the initial setup is rather low further compounding the efficacy of our proposed method.

\noindent \newline With that said, at the time of writing and testing, the classification step of our pipeline is definitely lacking -- an averaged accuracy of $\sim 71$, while not necessarily bad, is not anything to write home about and could cause issues down the line. Room for improvement lies both within the classifier used and the optimization process; however, we note that a lack of contextual information that serves to explain the emergence of the clusters as part of the clustering step could very well likely be the reason for obtaining sub-par accuracy scores. As it currently stands, the clustering step was built upon grouping together days that exhibited the highest similarity in terms of their energy consumption patterns. Given that this information is not available to us when considering a new day, we are left reaching for straws attempting to explain when any individual household is likely to observe energy consumption patterns that fall within any of the obtained clusters. Evidently, temporal and meteorological information is not enough to explain the emergence of said clusters and other information (perhaps patterns in terms of cluster labels leading up to the new sample) could serve to improve classifier accuracy. This is definitely an area of this study that could be looked into as part of future research.

\noindent \newline On the other hand, were we to disregard the shortcomings of the classification step of the pipeline, the overall improvements in forecasting accuracy that our \gls{cnn-lstm} network managed to achieve over other models and setups available in the current literature. Given enough time, further improvements can probably be made given changes to the overall network architecture and hyperparameter optimization. Overall, initial results seem quite promising and pave the way for further improvements to be made down the line, both in terms of forecasting accuracy as well as the overall structure of the entire pipeline and, in general, with regards to elaboration and increased clarity over each of the steps undertaken to achieve said results.