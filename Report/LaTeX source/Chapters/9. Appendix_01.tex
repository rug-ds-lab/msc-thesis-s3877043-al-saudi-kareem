%*******************************************************
% Appendix
%*******************************************************
\appendix
\chapter{Appendix}
\label{ch:Appendix}

\section{Figures}
\label{sec:Appendix:Figures}

\begin{figure}[H]
    \centering
    \includegraphics[width=0.75\textwidth]{Images/Chapter 5/REFIT/REFIT-House-12-Grangers-Causality-Matrix-Single.png}
    \caption{A trimmed subset of the Granger Causation matrix present in Figure \ref{fig:REFIT-House-12-Grangers-Causality-Matrix-All} that displays only the relevant information with regards to our independent variables causing our target variable.}
    \label{fig:REFIT-House-12-Grangers-Causality-Matrix-Single}
\end{figure}

\begin{figure}[H]
    \centering
    \includegraphics[width=0.75\textwidth]{Images/Chapter 5/UCID/UCID-Grangers-Causality-Matrix-Single.png}
    \caption{A trimmed subset of the Granger Causation matrix present in Figure \ref{fig:UCID-Grangers-Causality-Matrix-All} that displays only the relevant information with regards to our independent variables causing our target variable.}
    \label{fig:UCID-Grangers-Causality-Matrix-Single}
\end{figure}


\section{Tables}
\label{sec:Appendix:Tables}

\begin{table}[H]
        \begin{adjustwidth*}{-3.0cm}{-3.0cm}%
                \myfloatalign
                \centering
                \begin{tabularx}{\linewidth}{cX} \toprule
                        \tableheadline{Parameter}                & \tableheadline{Description}                                                                                                                                                                                                                               \\ \midrule
                        \gls{ghi}                                & The total irradiance received on a horizontal surface. It is the sum of the horizontal components of direct (beam) and diffuse irradiance. Units in W/m2.                                                                                                 \\
                        \gls{ebh}                                & The horizontal component of \glsentryfull{dni}. Units in W/m2.                                                                                                                                                                                            \\
                        \gls{dni}                                & Solar irradiance arriving in a direct line from the sun as measured on a surface held perpendicular to the sun.  Units in W/m2.                                                                                                                           \\
                        Zenith                                   & The angle between a line perpendicular to the earth's surface and the sun (90 deg = sunrise and sunset; 0 deg = sun directly overhead). Units in degrees.
                        \\
                        Azimuth                                  & The angle between a line pointing due north to the sun's current position in the sky. Negative to the East. Positive to the West. 0 at due North. Units in degrees.
                        \\
                        Cloud Opacity                            & The measurement of how opaque the clouds are to solar radiation in the given location. Units in percentage.                                                                                                                                               \\
                        Air Temperature                          & The air temperature (2 meters above ground level). Units in Celsius.                                                                                                                                                                                      \\
                        Dewpoint                                 & The air dewpoint temperature (2 meters above ground level). Units in Celsius.                                                                                                                                                                             \\
                        Relative Humidity                        & The air relative humidity (2 meters above ground level). Units in percentage.                                                                                                                                                                             \\
                        SFC pressure                             & The air pressure at ground level. Units in hPa.                                                                                                                                                                                                           \\
                        Wind Speed                               & The wind speed (10 meters above ground level). Units in m/s.                                                                                                                                                                                              \\
                        Wind Direction                           & The wind direction (10 meters above ground level). This is the meteorological convention. 0 is a northerly (from the north); 90 is an easterly (from the east); 180 is a southerly (from the south); 270 is a westerly (from the west). Units in degrees. \\
                        Preciptable Water                        & The total column preciptable water content. Units in kg/m2.                                                                                                                                                                                               \\
                        Snow Depth                               & The snow depth liquid-water-equivalent. Units in cm.                                                                                                                                                                                                      \\
                        \gls{gti} Horizontal Single-Axis Tracker & The total irradiance received on a sun-tracking surface.  Units in W/m2.                                                                                                                                                                                  \\
                        \gls{gti} Fixed                          & The total irradiance received on a surface with a fixed tilt. The tilt is set to latitude of the location.  Units in W/m2.                                                                                                                                \\
                        Albedo                                   & Average daytime surface reflectivity of visible light, expressed as a value between 0 and 1. 0 represents complete absorption. 1 represents complete reflection.                                                                                          \\
                \end{tabularx}
                \caption{List of meteorological parameters available to us as per the Solcast data sets as outlined in Section \ref{sec:Data-Description:Meteorological-Data}.}
                \label{tab:Solcast-parameters}
        \end{adjustwidth*}
\end{table}