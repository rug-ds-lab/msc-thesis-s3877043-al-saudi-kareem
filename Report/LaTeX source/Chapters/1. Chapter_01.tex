\chapter{Introduction}
\label{ch:Introduction}
Over the years, our reliance on electrical appliances has been slowly increasing (as shown in Figure \ref{fig:Appliance-Usage-Growth}). As our dependence on electrical appliances increases, so too does our consumption of energy \cite{WBCSD, Yixuan} and, subsequently, our need for more sophisticated and advanced solutions that can accommodate this growth. Thankfully, the convergence of multiple technologies -- such as machine learning, data mining and ubiquitous computing -- has led to the rise of a solution in the form of \textit{smart (electric) grids} as well as \textit{smart environments} and \textit{smart meters} that are slowly but surely taking off in terms of their popularity and availability \cite{Chao}. The resulting growth in the prevalence of smart grids gives us the opportunity to both control and monitor the energy consumption of individual households on a real-time basis \cite{Yildiz}, and, through the utilization of applications built upon this framework, we are capable of achieving an overall reduction in terms of the amount of energy that we, as the human race, consume. This opens up the possibility to alleviate some of the inherent risks associated with the growth in energy consumption, whether that be our overall environmental footprint on the planet or, on a much smaller scale, the financial impact on both suppliers as well as consumers due to instabilities present in current, outdated power grid systems \cite{Hsiao}.

\begin{figure}[hbt!]
    \centering
    \includegraphics[width=\textwidth]{Images/Chapter 1/Statista/Appliance-Usage-Growth.pdf}
    \caption{Historical/predicted growth in the number of appliances being used worldwide. Image source: \cite{Statista} © 2019, Statista.}
    \label{fig:Appliance-Usage-Growth}
\end{figure}

\noindent \newline Existing solutions developed under the increasingly popular smart grid framework, such as the \gls{hems} and \gls{bems}, aim to provide the end-user with the means to schedule, or otherwise manage, daily appliance operations, taking into consideration external factors such as weather conditions, utility tariff rates alongside any other personal preferences \cite{Yildiz}. To operate efficiently, these solutions rely on our ability to capably forecast future trends in energy consumption at the individual household level. This information is required to appropriately and sufficiently control and supply the correct energy load to the end-user \cite{Raza, Kareem}. This has lead to a shift in interest within the realm of load forecasting, in which prior research has predominantly been focused at the large-scale, regional level \cite{Foucquier} where an amalgamation of available data spanning numerous households provides more obvious patterns as a result of the underlying diversity between households being lost \cite{Kong} towards the individual household level. Furthermore, owing to the operational characteristics of both \gls{hems} and \gls{bems} and similar applications, load forecasting in the very short term (anywhere from a few minutes to a couple of hours), oftentimes referred to as \gls{vstlf}, are more relevant than the substantially studied longer term horizons that are predominantly associated with long-term network planning and operations \cite{Yildiz}.

\begin{figure}[hbt!]
    \centering
    \includegraphics[width=\textwidth]{Images/Chapter 1/Yang/HEMS-Architecture.pdf}
    \caption{The \gls{hems} architecture visualized. Image source: \cite{Yang} \citeIEEE{Yang}.}
    \label{fig:HEMS-Architecture}
\end{figure}

\noindent \newline When exploring energy consumption at the individual household level, the diversity and complexity associated with human behavior leads to extremely dynamic, volatile patterns that can prove to be highly dissimilar between households. In addition to this, certain households exhibit no clear pattern in energy consumption due to a high level of irregularity in the lifestyle of its occupants \cite{Kong}. To account for this dissimilarity, current, state-of-the-art methods benefit from a precursory clustering step within the forecasting pipeline \cite{Yildiz, Kong, Hsiao}. This precursory clustering step serves to amalgamate days that exhibit a measure of similarity in terms of their energy consumption patterns into the same cluster. By training individual forecasting models on a per-cluster basis we should, in theory, see an improvement in load forecasting performance as each of the respective models specializes in predicting future trends in energy consumption based on patterns present within the energy profile associated with its unique cluster. This is the area of research that this paper seeks to tackle -- how can we best construct energy profiles out of historical data that truly capture repeated patterns with regards to energy consumption and what are the effects of a clustering step in the performance of a forecasting pipeline.

\noindent \newline The following Chapters of this paper are organized as follows: Chapter \ref{ch:Related-Work} presents related work within the field of clustering and classifying energy profiles so as to establish a baseline with which to compare our work to. Following that, Chapter \ref{ch:Background-Information} serves to provide a brief, intuitive explanation of important concepts that are related to this paper. Chapter \ref{ch:Exploratory-Data-Analysis} will both describe as well as visualize the historical data that we have on hand, outlined in Section \ref{sec:Introduction:Introduction-to-the-Data}, for the duration of this project. Ensuingly, Chapter \ref{ch:Methodology} will outline our methodology with regards to both our chosen clustering as well as forecasting techniques. Finally Chapters \ref{ch:Results-and-Discussion} and \ref{ch:Conclusion-and-Future-Work} conclude the paper by presenting our results alongside a discussion and potential direction with regards to future work.

\section{Introduction to the Data}
\label{sec:Introduction:Introduction-to-the-Data}
At our disposal are a number of publicly available data sets that contain historical data with regards to energy consumption. These include the data collected by the \gls{epsrc} via the project entitled "\textit{\gls{refit}}" \cite{REFIT} which is a collaboration among the Universities of Strathclyde, Loughborough and East Anglia, as well as the \textit{"Individual Household Electric Power Consumption"} data set \cite{UCID} that is part of the \gls{uci} Machine Learning Repository and that will henceforth be acronymized as the \textit{"\gls{ucid}"}. This section will serve to briefly describe the main aspects of each of these individual data sets so that we may be better able to draw comparisons between them and highlight any key differences. Further in-depth analysis of each subsequent data set can be found in Chapter \ref{ch:Exploratory-Data-Analysis} of this paper. Additionally, we aim to append meteorological features (\eg temperature, wind speed, cloud coverage, precipitation) to each of our respective data sets -- an overview of this process and the data that we will be utilizing will also be presented in this section.

\subsection{REFIT}
\label{subsec:Introduction:Introduction-to-the-Data:REFIT}
The \gls{refit} Electrical Load Measurements data set \cite{REFIT} includes cleaned electrical consumption data, in watts, for a total of 20 households labeled \textit{House 1 - House 21} (skipping House 14) located in the Loughborough area, a town in England, over the period of 2013 through early 2015. The electrical consumption data is collected at both the aggregate level as well as the appliance level with each household containing a total of 10 power sensors that comprise of a current clamp for the household aggregate labeled as \textit{Aggregate} in the data set as well as 9 \glspl{iam} labelled as \textit{Appliance 1 - Appliance 9} in the data set. The appliance list associated with each of the \glspl{iam} differs between households and comprise a measure of ambiguity as applicants may have switched appliances around during the duration of the data collection and the installation team responsible for setting up the power sensors did not always collect relevant data associated with said \glspl{iam}. The consequences of this is of course that we do not know with 100\% certainty whether an appliance or set of appliances associated with an \gls{iam} is the same throughout the entirety of the data set. Additionally, some labels are inherently ambiguous taking, for example, the \textit{television site} label which could comprise of any number of appliances including: a television, DvD player, computer, speakers etc. Finally, the makes and models of the appliances that were meant to be collected by the installation team are not always present, further compounding on the previously mentioned uncertainties.

\noindent \newline The documentation associated with the data set states that active power is collected, and subsequently recorded, at an interval of 8 seconds; however, a cursory glance at the data demonstrates that this is not always the case. A potential reason for this could be the fact that the aforementioned power sensors are not synchronized with the associated collection script which polls within a range of 6 to 8 seconds leaving us with a margin for error in the intervals between recorded data samples. Moreover, the data set is riddled with long periods of missing data making it exceptionally difficult to work with. All of that said, the data collection team made an attempt to pre-process or otherwise \textit{clean} the data set by:

\begin{enumerate}
    \item Correcting the time to account for the \gls{uk} daylight savings.
    \item Merging timestamp duplicates.
    \item Moving sections of \gls{iam} columns to correctly match the appliance they were recording when said appliance was reset or otherwise moved.
    \item Forward filling \gls{nan} values or zeroing them depending on the duration of the time gap.
    \item Removing spikes of greater than 4,000 watts from the \gls{iam} values and replacing them with zeros.
    \item Appending an additional issues columns that is set to 1 if the sum of the sub-metering \glspl{iam} is greater than that of the household aggregate -- in this case, data should either be discarded or, at the very least, the discrepancy must be noted.
\end{enumerate}

\subsection{UCID}
\label{subsec:Introduction:Introduction-to-the-Data:UCID}
The \gls{ucid} data set \cite{UCID} contains a total of 2,075,259 measurements gathered in a single house located in Sceaux, a commune in the southern suburbs of Paris, France. The data within this data set was recorded throughout a duration of 47 months spanning the period between December 2006 and November 2010. Measurements were made approximately once a minute and consist of the minute-averaged active power consumption, in kilowatts, within the entire household as well as 3 energy sub-metering measurements that correspond to the kitchen, which includes a dishwasher and microwave, the laundry room that consists of a washing machine and tumble dryer, and the combination of both an electric water-heater as well as an air-conditioner respectively. The \gls{ucid} data set is not without fault either, containing approximately 25,979 missing measurements which make up roughly 1.25\% of the entire data set; however, given the extensive range covered as well as the immense number of total measurements available on hand these missing values can easily be disregarded and subsequently discarded during the preprocessing stage of our forecasting pipeline.

\subsection{Meteorological Data}
\label{subsec:Introduction:Introduction-to-the-Data:Meteorological-Data}
As an addendum to both the \gls{refit} and \gls{ucid} data sets we will be incorporating meteorological data as provided by Solcast \cite{Solcast}, a company based in Australia that aims to provide high quality and easily-accessible solar data. For the purpose of this master's thesis project we requested meteorological data in variable time resolutions (5, 10, 15 minutes) for both the Loughborough area in the \gls{uk} for the \gls{refit} data set as well as meteorological data for the Sceaux commune in the southern suburbs of Paris, France for the \gls{ucid} data set. The relevant periods are the 16th of September, 2013 up to and including the 11th of July, 2015 and the 1st of December, 2006 up to and including the 30th of November, 2010 for each data set respectively. The provided data is extensive, covering a wide range of parameters that are listed, and described in detail, in Table \ref{tab:Solcast-parameters} which is located in Appendix \ref{ch:Appendix:Tables} of this paper.

\section{Proposed Model}
\label{sec:Introduction:Propose-Model}
To attempt to solve the previously outlined problem of \gls{vstlf} forecasting at an individual household level we propose a novel solution that utilizes a combination of statistical knowledge and machine learning techniques to both generate energy profiles that provide us with some measure of insight as to the habits of a household's occupants as well as forecast future trends in their energy consumption. A high level overview of the steps relevant to our proposed model can be seen in Figure \ref{fig:Proposed-Model-1}.

\null \vspace{0.5em}

\begin{figure}[H]
    \begin{adjustwidth}{-3.0cm}{-3.0cm}%
        \centering
        \includegraphics[width=\linewidth]{Images/Chapter 1/Other/High Level Model.pdf}
        \caption{High level overview of the steps pertaining to our model.}
        \label{fig:Proposed-Model-1}
    \end{adjustwidth}
\end{figure}

\noindent \newline In short, we devised a method that consists of 3 steps: cluster, classify, forecast -- first we cluster historical days based on similarity in terms of active power consumption, then we classify new days into one of the generated clusters and finally we generate forecasts based on models that are trained on a per-cluster basis. These steps, alongside a working example, will be discussed in-depth in Chapter \ref{ch:Methodology}.