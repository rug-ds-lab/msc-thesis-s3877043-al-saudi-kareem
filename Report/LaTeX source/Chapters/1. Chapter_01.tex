\chapter{Introduction}
\label{ch:Introduction}
Our reliance on energy is one that is ever-increasing now, more-so than ever. As our dependence on electrical appliances continues to grow over the years \cite{Statista, WBCSD, Yixuan} so too does our need for smarter, more sophisticated and advanced power grids. Thankfully, the convergence of multiple technologies -- the likes of machine learning, data mining and ubiquitous computing has led to the rise of a solution in the form of \textit{smart (electric) grids} as well as \textit{smart environments} that are slowly but surely taking off in terms of their popularity and availability \cite{Chao}. The resulting growth in the prevalence of smart grids gives us the opportunity to both control and monitor the energy consumption of individual households on a real time usage basis \cite{Yildiz} leading to an increase in efficiency and subsequently, an overall reduction in terms of the amount of energy we, as the human race, consume. This opens up the possibility to alleviate some of the inherent risks associated with the growth in energy consumption whether that be our overall environmental footprint on the planet or, on a much smaller scale, the financial impact on both suppliers as well as consumers due to instabilities present in current, outdated power grid systems \cite{Hsiao}.

\begin{figure}[hbt!]
    \centering
    \includegraphics[width=\textwidth]{Images/Chapter 1/Statista/Appliance-Usage-Growth.PNG}
    \caption{Historical/predicted growth in the number of appliances being used worldwide. Image source: \cite{Statista} © 2019, Statista.}
    \label{fig:Appliance-Usage-Growth}
\end{figure}

\noindent \newline Applications developed under the increasingly popular smart grid framework provide us with the means to such an end. Existing solutions such as the \gls{hems} and \gls{bems} aim to provide the end-user with the means to schedule, or otherwise manage, daily appliance operations taking into consideration external factors such as weather conditions, utility tariff rates as well as personal preference \cite{Yildiz}. These solutions rely on the ability to capably predict or, in other words, \textit{forecast} future trends in energy consumption \cite{Kareem} so as appropriately and sufficiently control and supply the correct energy load to the end-user \cite{Raza}. The concept of load forecasting is far from novel having been extensively studied within the literature \cite{Foucquier}; however, the majority of studies focus on load forecasting on the large-scale, regional level where an amalgamation of available data spanning numerous households provides more consistently obvious patterns as a result of the underlying diversity between households being lost when taking the aggregated residential level \cite{Kong}.

\begin{figure}[hbt!]
    \centering
    \includegraphics[width=\textwidth]{Images/Chapter 1/Yang/HEMS-Architecture.PNG}
    \caption{The \gls{hems} architecture visualized. Image source: \cite{Yang} \citeIEEE{Yang}.}
    \label{fig:HEMS-Architecture}
\end{figure}

\noindent \newline When exploring energy consumption at the individual household level, the diversity and complexity associated with human behaviour leads to extremely dynamic, volatile patterns that can prove to be highly dissimilar between households. In addition to this, certain households exhibit no clear pattern in energy consumption due to a high level of irregularity in the lifestyle of its occupants \cite{Kong}. To account for this dissimilarity current, state-of-the-art methods generally benefit from a precursory clustering step within the forecasting pipeline \cite{Yildiz, Kong, Hsiao} and this is the area of research that this paper seeks to tackle -- how can we best construct energy profiles out of historical data and what are the effects of a clustering, or otherwise, classification step in the performance of a forecasting pipeline.

\noindent \newline The following chapters of this paper will be organised as follows: chapter \ref{ch:Related-Work} will present related work within the field of clustering and classifying energy profiles so as to establish a baseline with which to compare our work to. Following that chapter \ref{ch:Background-Information} serves to provide a brief, intuitive explanation of the concepts related to this paper -- this explanation will be fairly high-level in the name of preserving time and preventing the paper from being lengthy. Chapters \ref{ch:Data-Description} and \ref{ch:Exploratory-Data-Analysis} will both describe as well as visualize the historical data that we have on hand for the duration of this project. Ensuingly, chapter \ref{ch:Methodology} will outline our methodology with regards to both our chosen clustering as well as forecasting techniques. Finally chapters \ref{ch:Results-and-Discussion} and \ref{ch:Conclusion-and-Future-Work} conclude the paper by presenting our results alongside a discussion and potential direction with regards to future work.